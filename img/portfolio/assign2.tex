\documentclass[11pt]{article}
\usepackage{amsthm}
\usepackage{fullpage}
\usepackage{epsfig}
\usepackage{amsmath}
\usepackage{amsfonts}
\usepackage[official]{eurosym}
\usepackage[]{algorithm2e}
\usepackage[normalem]{ulem}
\usepackage{color}
\usepackage{tcolorbox}


\newcommand{\remove}[1]{{}}
\newcommand{\ignore}[1]{{}}
\newcommand{\rednote}[1]{{\color{red} #1}}
\newcommand{\bluenote}[1]{{\color{blue} #1}}



\begin{document}

\noindent
{CS 341, Fall 2017 \hfill  A. Lubiw, B. Ma, E. Zima}


\begin{center}
{\large\bf ASSIGNMENT 2}
\end{center}

\bigskip
\noindent
DUE: Wednesday September 27, 7 PM.
DO NOT COPY.  ACKNOWLEDGE YOUR SOURCES.

\medskip
\noindent
Please read \verb| http://www.student.cs.uwaterloo.ca/~cs341 | for general
instructions and policies.

\medskip
\noindent
NOTE: There is a programming question in a separate file.


\medskip
\noindent

\begin{enumerate}

%%%%%%%%%%%%%%%%%%%%%%%%%%%%%%%%%%%%
\item{[10 marks]}  {\bf Divide-and-conquer.}
One popular way to rank researchers is by their  ``$h$-index''.  
A researcher's $h$-index is the maximum integer $k$ such that the researcher has at least $k$ papers that have been cited at least $k$ times each.
Suppose Professor X has written $n$ papers and paper $i$ has been cited $c_i$ times.  
Suppose you have these sorted in an array $C$ with $c_1 > c_2 > \ldots > c_n$.  
Give a divide-and-conquer algorithm to find Professor X's $h$-index.  Your algorithm should run in time $O(\log n)$.  

As noted on the course web page, ``giving'' an algorithm means: describe the algorithm briefly in words, give high-level pseudocode, justify correctness, and analyze run-time.


%%%%%%%%%%%%%%%%%%%%%%%%%%%%%%%%%%%
\item{[15 marks]} {\bf Squaring a matrix.}
\begin{enumerate}
\item{} [3 marks] Suppose you are given a $2 \times 2$ matrix $A$ and you want to compute $A^2$.  Show that you can do this with 5 multiplications.


\item{} [6 marks]  Consider the following divide-and-conquer algorithm to compute $A^2$ when $A$ is an $n \times n$ matrix:  The algorithm is like Strassen's algorithm except that we get 5 problems of size $n/2$ (by part (a)) instead of getting 7 subproblems of size $n/2$ as in Strassen's algorithm.  This gives a run-time of $O(n^{\log_2 5}) \approx O(n^ {2.32})$---even better than Strassen's run-time of $O(n^{2.81})$.  

Explain what is wrong with the above algorithm and analysis. 


\item{} [6 marks]  Show that squaring matrices is in fact no easier than multiplying matrices.  To do this, prove that if there is an algorithm with run time $O(n^c)$ that will square an $n \times n$ matrix, then there is an algorithm with run-time $O(n^c)$ that will multiply two $n \times n$ matrices.
(In other words, you will be reducing the problem of multiplying matrices to the problem of squaring matrices.)   


\end{enumerate}

\end{enumerate}


\end{document}
